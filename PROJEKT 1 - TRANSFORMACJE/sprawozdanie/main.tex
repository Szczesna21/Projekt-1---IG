\documentclass[10pt,a4paper]{article}

% ------------------------- PREAMBUŁA -------------------
\input{settings/packages.tex}   % ścieżka względna do katalogu z ustawieniami
\graphicspath{{images/}}    % stała ścieżka względna do katalogu z  obrazkami.



% METADATA
% DOCUMENT METADATA
\newcommand{\logoGIK}{settings/WGiK-znak.png}
\newcommand{\authorName}{Julia Szczęsna \\ grupa 1, Numery Indeksu: 312035}
\newcommand{\titeReport}{Projekt 1} % <<< here insert short title in the food
\newcommand{\titleLecture}{informatyka Geodezyjna\\ sem. IV, ćwiczenia, rok akad. 2021-2022} % <<< insert title of presentation
\newcommand{\kind}{report}
%\newcommand{\mymail}{\href{mailto:01160065@pw.edu.pl}{01160065@pw.edu.pl}}
\newcommand{\supervisor}{....}
\newcommand{\gikweb}{\href{www.gik.pw.edu.pl}{www.gik.pw.edu.pl}}
\newcommand{\institut}{Zakład Geodezjii i Astronomii Geodezyjnej}
\newcommand{\faculty}{Wydział Geodezji i Kartografii}
\newcommand{\university}{Politechnika Warszawska}
\newcommand{\city}{Warszawa}
\newcommand{\thisyear}{2022}
%\date{}
% PDF METADATA
\pdfinfo
{
	/Title       (GIK PW)
	/Creator     (TeX)
	/Author      (Julia Szczęsna)
}


% ------------------------- POCZATEK DOKUMENTU -------------------
\begin{document}
	% ----------------------------------------------------------------
	% ----------------------------  Title page
	% ----------------------------------------------------------------
	\begin{center} 
		\rule{\textwidth}{.5pt} \\
		\vspace{1.0cm}
		\includegraphics[width=.4\paperwidth]{\logoGIK}
		\vspace{0.5cm} \\
		\Large \textsc{\titeReport}
		\vspace{0.5cm} \\  
		\large \textsc{\titleLecture}
		\vspace{0.5cm}\\
		\textsc{\authorName}  \\
		\textsc{\faculty}, \textsc{\university}  \\ 
		\city, \today
	\end{center} 
	\rule{\textwidth}{1.5pt}
	
	
	\newpage
	% ---------------------------------------------------------------
	% ----------------------------  Table of content
	% ----------------------------------------------------------------
	\tableofcontents 								% wyświetla spis treści
	%\addcontentsline{to}{chapter}{Spis treści} 	% dodaje pozycję do spisu treści
	% \listoffigures  								% wyświetla spis rysunków
	%\addcontentsline{toc}{chapter}{Lista rysunków} % dodaje pozycję do spisu treści
	% \listoftables 									% wyświetla spis rysunków
	%\addcontentsline{toc}{chapter}{lista tabel}	% dodaje pozycję do spisu treści
	\newpage
	%
	
	
	% ----------------------------------------------------------------
	% ----------------------------  Sekcje i podsekcje  
	% ----------------------------------------------------------------
	
	
	\section{Założenia do wykonania projektu.}
Informacje dotyczące przygotowania aplikacji:
\begin{itemize}
	\item Aplikacja powinna zawierać podstawowe algorytmy tranformacji:

	\subitem współrzędnych geocentrycznych (X,Y,Z) na współrzędne geodezyjne ($\phi, \lambda, h$)
	\subitem współrzędnych geodezyjnych ($\phi, \lambda, h$) na współrzędne geocentryczne (X,Y,Z)
	\subitem wyznaczenia współrzędnych topocentrycznych (N, E,U)
	\subitem wyznaczenie współrzędnych w układzie 2000
	\subitem wyznaczenie współrzędnych w układzie 1992
	\subitem wyznaczenie kąta azymutu i kąta elewacji
	\subitem wyznaczenie odległości 2D i 3D
	
	
	\item Aplikacja powinna byc napisana w Klasie Pythona;
	
	\item  Kolejne etapy tworzenia aplikacji powinny być utrzymane w systemie kontroli wersji git;
	
	\item  Sprawozdanie techniczne należy przygotowac w LaTEX;
	
	\item Sprawozdanie powinno zawierać krótki opis zadania oraz link do github z kodem programu;
\end{itemize}  
	\newpage      
	\section{Instrukcja obsługi programu.}
Program daje możliwość na początku wyboru elipsoidy, na której mają zostać wykonane obliczenia. 
\begin{itemize}
	\item Pierwsza
	\item Druga
\end{itemize}
Program, pobierając dane odpowiedniej elipsoidy, wykonuje wybrane przez użytkownika transformacje. 
\newline
Użytkownik ma do wyboru transformacje:
\begin{itemize}
	\item współrzędnych geocentrycznych (X,Y,Z) na współrzędne geodezyjne ($\phi, \lambda, h$)
	\item współrzędnych geodezyjnych ($\phi, \lambda, h$) na współrzędne geocentryczne (X,Y,Z)
	\item wyznaczenia współrzędnych topocentrycznych (N, E,U)
	\item wyznaczenie współrzędnych w układzie 2000
	\item wyznaczenie współrzędnych w układzie 1992
	\item wyznaczenie kąta azymutu i kąta elewacji
	\item wyznaczenie odległości 2D 
	
\end{itemize}
Program wykorzystuje do obliczeń transformacje zawarte w odpowiednich plikach projektu. Wywołane przez odpowiedzialne za to komendy. 
\newline
Do każdego rodzaju transformacji dołączona została informacja o wykorzystaniu danej funkcji oraz jej parametrach wyjściowych i wejściowych. 


	\newpage
	\section{Link do programu.}
Poniższy link umożliwia przejście do folderu github'a, gdzie znajduje się plik z kodem aplikacji.
\url{https://github.com/Szczesna21/Projekt-1.git}
	
	
	
	
	
	
	% ----------------------------------------------------------------
	% ----------------------------  Bibliography  
	% ----------------------------------------------------------------
	
\end{document}
