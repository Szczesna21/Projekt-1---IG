\section{Instrukcja obsługi programu.}
Program daje możliwość na początku wyboru elipsoidy, na której mają zostać wykonane obliczenia. 
\begin{itemize}
	\item Pierwsza
	\item Druga
\end{itemize}
Program, pobierając dane odpowiedniej elipsoidy, wykonuje wybrane przez użytkownika transformacje. 
\newline
Użytkownik ma do wyboru transformacje:
\begin{itemize}
	\item współrzędnych geocentrycznych (X,Y,Z) na współrzędne geodezyjne ($\phi, \lambda, h$)
	\item współrzędnych geodezyjnych ($\phi, \lambda, h$) na współrzędne geocentryczne (X,Y,Z)
	\item wyznaczenia współrzędnych topocentrycznych (N, E,U)
	\item wyznaczenie współrzędnych w układzie 2000
	\item wyznaczenie współrzędnych w układzie 1992
	\item wyznaczenie kąta azymutu i kąta elewacji
	\item wyznaczenie odległości 2D 
	
\end{itemize}
Program wykorzystuje do obliczeń transformacje zawarte w odpowiednich plikach projektu. Wywołane przez odpowiedzialne za to komendy. 
\newline
Do każdego rodzaju transformacji dołączona została informacja o wykorzystaniu danej funkcji oraz jej parametrach wyjściowych i wejściowych. 

